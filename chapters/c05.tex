\chapter{Conclusion} 
\label{Conclusion}

The main goal of this work was to test the resistance of regular and perfect OAM-induced-vortices to central obstructions, in order to pursue or discard FSO setups using Schmidt-Cassegrain telescopes. These type of telescopes are commonly used in view of their advantages over others; additionally, OAM beams, which already have a central dark region within them, seemed like a perfect candidate to test the impact of central obstructions on what appears to be already-obstructed beams. The experiments were simulated on MATLAB and showed that perfect vortices are substantially more resistant to deformation caused by central obstructions than regular ones.

To corroborate this, the simulation comprehended the creation, obstruction, propagation and analysis of both types of vortices. Their structure, that is their dark central region, was confirmed to be affected by taking linear intensity profiles of the real field. Simultaneously, their apparent topological charge was measured qualitatively by taking a circular profile around the phase mask's center. Finally, from the comparisons using different obstruction sizes and propagation distances, among other parameters, we can takeaway that:

\begin{enumerate}
    \item Perfect vortices are more resistant than regular ones to central obstructions, regardless of the obstruction's size.
    \item If long propagations --- with obstructions --- are to take place, prefer perfect vortices over regular ones for the most reliability.
    \item However, their phase masks are too complex and, thereupon, hard to use. So, if phase masks are critical to the application, consider using regular vortices and shortening the propagation distance. If it is not possible, reconsider whether if an optical link is appropriate at all.
    \item Besides, perfect vortices' intensity can drop beyond 50\% with large obstructions, and thus could be negatively affected for very long propagations where intensity is at stake.
\end{enumerate}

To further develop this work and boost its outcomes, phase cleaning methods should be applied - do consider that they are computationally expensive. Now, because the topological charge measurements shown here are qualitative, they could be misinterpreted regarding the ``true'' topological charge. Also consider that obstructing the phase mask produces a composite field between a Gaussian, the obstruction and/or remnants of the original phase mask. In conclusion, the results obtained are promising, and lay the foundations for future works to help develop more robust FSO links that could eventually compete with current communication systems.